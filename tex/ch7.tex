\section{Conclusion}
Spin ice structures have no definite application as of yet but the most likely procurement from studying these structures is that it may be possible to use them in future spintronic devices. They could potentially be used as high density data storage devices. W.R. Branford et al. recently demonstrated the ability to 'read' and 'write' patterns in a honeycomb lattice. {\cite{b14}}
\par
Even given current memory standards of approximately 1.73$\times 10^{5}$ bits per mm$^{2}$ in domestic computers,  a spin ice like the one used by W.R. Bradford et al. could potentially lead to memory devices with storage densities of order 1 $\times 10^{6}$ bits per mm$^{2}$.
\par
Another interesting application is in the area of logic devices. The frustration in spin ice could be harnessed to perform complex computational problems in a single calculation, by using it as a kind of neural network.
\par
In the string collisions we see evidence of magnetic charge behaviour, with attraction between opposite charges and repulsion between same sign charges. Furthermore, the simulations show that the positive charges move in the same direction as the external magnetic field and the negative charges move in the opposite direction.
\par
This is analogous to a semiconductor in which the positively charged holes move in the direction of an external electric field and the negatively charged electrons move opposite to the external field. Thus we can say that the charge model seems to be a valid model in the case of the square lattice.  The algorithms and Monte Carlo used in this project were adapted from the kagome lattice.
\par
Due to constraints on computational resources available the data set is limited so other interesting questions requiring a statisically significant set of data were not explored. As an extension, it would be interesting to examine the avalanche behaviour of the system could be examined in more detail, with the form of the avalanche probability P(s) (probability of an avalanche as a function of the number of spins in the avalanche) still to be investigated.
\par
Nonetheless this project has yielded a better insight into the behaviour of the frustrated Kagome lattice with evidence for the emergence of fractional quasi-particles or `magnetic monopoles' and Dirac strings in 2-D artificial spin ice.
\clearpage

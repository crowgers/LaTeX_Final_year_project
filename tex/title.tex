\title{Emergent Fractional Quasi-particles in \\ Geometrically Frustrated Systems}
\author{Cormac Rogers\\09427759}
\date{December, 2014}

\begin{titlingpage}
    \begin{center}
        \includegraphics[scale=0.3]{UCDlogo.png}\\
        \vspace{1.5cm}
        \begin{LARGE}
            \textbf{\thetitle}\\
        \end{LARGE}
        \vspace{1.5cm}
        \theauthor\\
        \vspace{1.5cm}
        This report is submitted to University College Dublin in part fulfilment of the requirement for the degree of Bachelor of Science in Theoretical Physics.\\
        \vspace{1.5cm}
        \textbf{Supervisor:} Prof. Hans Bajamin-Braun\\UCD School of Physics\\
        \vspace{1.5cm}
        \thedate
        \begin{abstract}
            \noindent Magnetic monopoles have been theoretically predicted to exist as elementary particles yet have not been observed in free space to this day. However, in a group of geometrically frustrated magnetic materials known as spin ice, emergent quasi-particles have been observed which behave as magnetic monopoles. Performing simulations on a kagome lattice, a corner-shared triangular lattice using a Monte-Carlo procedure, I was able to obtain evidence for these quasi-particles and their associated strings which were intrinsic properties of the lattice. Furthermore, it was found that these `monopoles/anti-monopoles' interact via a magnetic analogy to the coulomb interaction with same charges repelling each other and opposite charges attracting one another and annihilating.
        \end{abstract}
    \end{center}
\end{titlingpage}
